\documentclass{report}
\usepackage[utf8]{inputenc}

\title{1819-108-C4-W1-01}
\author{Kārlis Zariņš}
\date{May 19}

\begin{document}

\maketitle

\section*{Intro}

Tas ir mans pirmais dokuments ar LaTex. Man labi sanak rakstit!!!Tas ir mans pirmais dokuments ar LaTex. Man labi sanak rakstit!!!Tas ir mans pirmais dokuments ar LaTex. Man labi sanak rakstit!!!Tas ir mans pirmais dokuments ar LaTex. Man labi sanak rakstit!!!Tas ir mans pirmais dokuments ar LaTex. Man labi sanak rakstit!!!Tas ir mans pirmais dokuments ar LaTex. Man labi sanak rakstit!!!Tas ir mans pirmais dokuments ar LaTex. Man labi sanak rakstit!!!Tas ir mans pirmais dokuments ar LaTex. Man labi sanak rakstit!!!Tas ir mans pirmais dokuments ar LaTex. Man labi sanak rakstit!!!Tas ir mans pirmais dokuments ar LaTex. 

Tas ir mans pirmais dokuments ar LaTex. Man labi sanak rakstit!!!Tas ir mans pirmais dokuments ar LaTex. Man labi sanak rakstit!!!Tas ir mans pirmais dokuments ar LaTex. Man labi sanak rakstit!!!Tas ir mans pirmais dokuments ar LaTex. Man labi sanak rakstit!!!Tas ir mans pirmais dokuments ar LaTex. Man labi sanak rakstit!!!Tas ir mans pirmais dokuments ar LaTex. Man labi sanak rakstit!!!Tas ir mans pirmais dokuments ar LaTex. Man labi sanak rakstit!!!Tas ir mans pirmais dokuments ar LaTex. Man labi sanak rakstit!!!Tas ir mans pirmais dokuments ar LaTex. Man labi sanak rakstit!!!

\section{Discussion}

A. Einstein ir formulejis sekojoso sakaribu: $E=mc^2$

A. Einstein ir formulejis sekojoso sakaribu: $$E=mc^2$$

Mendelejevs ir formulejis VODKA: $C_2H_5OH+ H_2O$

Mendelejevs ir formulejis VODKA:$$C_2H_5OH+ H_2O$$

$$S(x)=\sum_{i=1}^\infty(1+\frac{x}{i^2})$$


\section{Conclusion}
End game

\end{document}
